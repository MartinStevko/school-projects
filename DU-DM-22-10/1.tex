%%%%%%%%%%%%%%%%%%%%%%%%%%%%%%% Riesenie %%%%%%%%%%%%%%%%%%%%%%%%%%%%%%%

Na to aby sme výrok vyvrátili, stačí nájsť prípad kedy máme čiastočné 
usporiadanie $\preceq$ a lineárne usporiadanie $\leq$ a zároveň 
$\preccurlyeq$ nie je lineárne usporiadanie. Zvoľme si teda rôzne 
$x$, $y$ a $z$ také, že $x\prec y$, $y\prec z$, no $x$ a $z$ sú 
v usporiadaní $\preceq$ neporovnateľné. Zároveň nech platí $z\leq x$. 
Keďže usporiadanie $\preceq$ má byť len čiastočné a usporiadanie 
$\leq$ lineárne, takáto podmnožina im zjavne vyhovuje. 

Potom však platí $x\preccurlyeq y\preccurlyeq z$ (pretože 
$x\prec y\prec z$). Pre spor predpokladajme, že usporiadanie 
$\preccurlyeq$ bude lineárne, potom z definície lineárneho usporiadania 
viem, že aj $x\preccurlyeq z$. Zároveň však keďže $x$ a $z$ sú 
v usporiadaní $\preceq$ neporovnateľné a $z\leq x$, tak 
$z\preccurlyeq x$, čo je spor s tým, že usporiadanie je lineárne, teda 
antisymetrické, keďže $x$ a $z$ som si zvolil ako rôzne. Usporiadanie 
$\preccurlyeq$ preto nemusí byť stále lineárne.

%%%%%%%%%%%%%%%%%%%%%%%%%%%%%%%%%%%%%%%%%%%%%%%%%%%%%%%%%%%%%%%%%%%%%%%%
