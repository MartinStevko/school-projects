%%%%%%%%%%%%%%%%%%%%%%%%%%%%%%% Riesenie %%%%%%%%%%%%%%%%%%%%%%%%%%%%%%%

Na to aby sme výrok vyvrátili, stačí nájsť prípad kedy máme čiastočné 
usporiadanie $\preceq$ a lineárne usporiadanie $\leq$ a zároveň 
$\preccurlyeq$ nie je lineárne usporiadanie. Zvoľme si teda rôzne 
$x$, $y$ a $z$ také, že $x\prec y$, no $y$, $z$ a  $x$, $z$ sú po
dvojiciach v usporiadaní $\preceq$ neporovnateľné. Zároveň nech platí 
$y\leq z$ a $z\leq x$. Keďže usporiadanie $\preceq$ má byť len 
čiastočné a usporiadanie $\leq$ lineárne, takáto podmnožina im zjavne 
vyhovuje. 

Potom však platí $y\preccurlyeq z\preccurlyeq x$ (pretože 
$y\leq z\leq x$). Pre spor predpokladajme, že relácia $\preccurlyeq$ 
bude lineárnym usporiadaním. Potom z definície lineárneho usporiadania 
viem, že aj $y\preccurlyeq x$. Zároveň však keďže $x\preceq y$, tak 
$x\preccurlyeq y$, čo je spor s tým, že usporiadanie je lineárne, teda 
antisymetrické, keďže $x$ a $y$ som si zvolil ako rôzne. Usporiadanie 
$\preccurlyeq$ preto nemusí byť stále usporiadaním, alebo aspoň nie 
lineárnym.

%%%%%%%%%%%%%%%%%%%%%%%%%%%%%%%%%%%%%%%%%%%%%%%%%%%%%%%%%%%%%%%%%%%%%%%%
