%%%%%%%%%%%%%%%%%%%%%%%%%%%%%%% Riesenie %%%%%%%%%%%%%%%%%%%%%%%%%%%%%%%

Pozrime sa na to, čo znamená ľavá strana. Najprv vyberiem $m$--prvkové 
podmnožiny z $n$--prvkovej množiny a ich počet vynásobím počtom 
$r$--prvkových podmnožín, ktoré pre každú z nich existujú. Dostanem 
teda číslo, ktoré znamená počet $r$--prvkových podmnožín z $n$--prvkovej 
množiny, pričom každá táto podmnožina tam bude zarátaná $x$--krát (lebo 
tú istu podmnožinu dostanem z viacerých $m$--prvkových podmnožín). Každá 
z nich tam je ale zarátaná $\binom{n-r}{m-r}$--krát pretože je to to 
isté, ako by som si $r$ prvkov vybral dopredu a potom zo zvyšných $n-r$ 
vyberal zvyšných $m-r$ prvkov. Dostávam teda 
\[\binom{n}{m} \cdot \binom{m}{r} = 
\binom{n}{r} \cdot \binom{n-r}{m-r} \] 
čo som mal dokázať.

%%%%%%%%%%%%%%%%%%%%%%%%%%%%%%%%%%%%%%%%%%%%%%%%%%%%%%%%%%%%%%%%%%%%%%%%
