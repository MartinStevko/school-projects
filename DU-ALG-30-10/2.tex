%%%%%%%%%%%%%%%%%%%%%%%%%%%%%%% Riesenie %%%%%%%%%%%%%%%%%%%%%%%%%%%%%%%

Maximálny počet banánov, ktoré vieme dopraviť do kilometra $d$ oznažme 
$v_d$. Tvrdím, že do kilometra $d+1$ vieme dopraviť najviac 
$$v_{d+1} = v_d - (2\cdot\lceil\dfrac{v_d-2S}{N}\rceil -1)\cdot S$$
banánov. Tvrdenie platí, pretože nech mám prenesených koľkokoľvek 
banánov, vždy ak na kôpke o kilometer dozadu ostalo aspoň $2S+1$ 
banánov, tak sa mi po ne oplatí vrátiť, pretože po ceste tam a späť zje 
velblúd iba $2S$ banánov. Potom viem vzťah upraviť všeobecne na:
$$v_d = V-\sum_{i = 0}^{d-1}(2\cdot\lceil\dfrac{v_d-2S}{N}\rceil -1)\cdot S$$
V úlohe v bode 2 mi teda stačí nájsť $v_D$ a mám riešenie, respektíve 
$v=v_D$.

V bode 1 mám zadané $v$ a potrebujem zistiť $D$. Vieme teda, že 
$$\dfrac{V-v}{S} = \sum_{i = 0}^{D-1}(2\cdot\lceil\dfrac{v_D-2S}{N}\rceil -1)$$
z čoho sa $D$ dá zistiť postupným pričítavaním ďalšieho prvku na pravú 
stranu, až kým nedostanem rovnosť. Potom som našiel hľadané $D$. Pre 
výpočet človekom je rovnica vyššie pomerne komplikovaná a dá sa 
zjednodušiť napríklad tak, že si všimneme, že prvky sumy sa niekoľkokrát 
za sebou opakujú (majú rovnakú hodnotu), takže by nám stačilo pozerať sa 
na hodnoty keď sa to zmení. Konkrétne by to pre bod 1 vyzeralo tak, že 
vyrátame $n$ -- v mojom riešení poslednú pozíciu zmeny ako 
$$n=\lceil\frac{V}{N}\rceil -\lceil\frac{v}{N}\rceil$$
a následne zrátam dĺžku celého úseku $d_n$ bez zmeny plus k nemu 
pripočítam sumu dĺžok predošlých úsekov, teda rovnica by vyzerala ako 
(aj s ošetrením všetkých podmienok aby sme nešli do záporu):
\[D = d_n = \frac{\min\{N, \max [(\lceil\frac{V}{N}\rceil -n+1)\cdot N, 
0]\}}{(2\cdot\frac{\max [(\lceil\frac{V}{N}\rceil -n+1)\cdot N, 
0]}{N}-1)\cdot S}+\sum_{i = 1}^{n-1}  d_i\]

Pre konkrétne hodnoty v zadaní teda $533$ kilometrov. 

%%%%%%%%%%%%%%%%%%%%%%%%%%%%%%%%%%%%%%%%%%%%%%%%%%%%%%%%%%%%%%%%%%%%%%%%
