%%%%%%%%%%%%%%%%%%%%%%%%%%%%%%% Riesenie %%%%%%%%%%%%%%%%%%%%%%%%%%%%%%%

Bez ujmy na všeobecnosti predpokladajme, že byty sú očíslované v smere 
hodinových ručičiek. Pre každú rodinu si poďme vyrátať ich vzdialenosť 
v smere hodinových ručičiek od miesta, kam sa chcú presťahovať (alebo 
o koľko bytov ich musíme posunúť). Na začiatku nech to je $1$ pre všetky 
rodiny, teda spolu $n$. Pozrime sa ale na to, že táto vzdialenosť sa 
nezmení ani ak vymením nejakú dvojicu, pretože 1 rodine sa vzdialenosť 
o 1 zníži a tej druhej zas o 1 zvýši. 

Premyslime si ale ešte ako to bude vyzerať v okolí miesta, kde nejaká 
rodina chce bývať, pretože uvažujeme vzdialenosti v jednom smere, takže 
tu sa budú výrazne meniť. Ak sa presťahujú do cieleného bytu, pričom 
sa sťahovali v smere hodinových ručičiek, tak ich vzdialenosť sa zníži 
o 1 a vzdialenosť druhej sťahovanej rodini sa o 1 zvýši, alebo nastane 
prípad, kedy sa aj táto druhá rodina presťahovala do bytu kde chce 
bývať, no tentokrát sa sťahovali v protismere hodinových ručičiek, čiže 
zo vzdialenosti $n-1$ sa stala vzdialenosť $0$. Potom súčet 
vzdialeností sa buď zachoval, alebo ak nastal prípad, kedy sa obe 
rodiny presťahovali tam, kde chcú bývať, tak sa súčet znížil o $n$. 
Ostáva sa ešte zamyslieť nad prípadom kedy sa rodina do cieleného bytu 
presťahuje v protismere hodinových ručičiek a druhá sťahovaná rodina sa 
do cieleného bytu nepresťahuje, no to je analogické ako prípad vyššie 
a súčet sa pri tom nezmení a nad prípadom, kedy sa rodina z cieleného 
bytu presťahuje v smere hodinových ručičiek o 1 ďalej, avšak vtedy 
nastáva prípad symetrický tomu popísanému vyššie a teda súčet sa o $n$ 
zvýši.

Na začiatku teda máme vzdialenosť $1$ pre každú rodinu a potrebujeme 
z nej spraviť nejaký násobok $n$, pretože až vtedy budú všetky rodiny 
v bytoch kde chcú. Potom sa ale každá rodina musí presťahovať aspoň raz 
v smere hodinových ručičiek, alebo aspoň $n-1$ krát v protismere 
hodinových ručičiek. Zároveň viem, že prípad, kedy sa každá rodina 
presťahuje iba aspoň raz v smere hodinových ručičiek nastať nemôže, 
lebo to by sme celkový súčet znížili na $0$ bez toho, aby sa nejaká 
rodina presťahovala do svojho bytu v protismere hodinových ručičiek, 
o čom sme dokázali, že sa to nedá. Musí teda nastať prípad, kedy sa 
nejaká rodina sťahuje aspoň $(n-1)$--krát, čo znamená, že za menej ako 
$n-1$ dní sa sťahovanie uskutočniť nedá. To že za $n-1$ dní sa to dá 
dokážem nájdením konkrétnej konštrukcie, napríklad tej, že zoberiem 
rodinu $n$ a prvý deň ju vymením s rodinou $n-1$, druhý deň s $n-2$, 
až $(n-1)$. deň s rodinou $n-(n-1)=1$. Potom budú všetky rodiny 
v bytoch kde byť chcú.

%%%%%%%%%%%%%%%%%%%%%%%%%%%%%%%%%%%%%%%%%%%%%%%%%%%%%%%%%%%%%%%%%%%%%%%%
