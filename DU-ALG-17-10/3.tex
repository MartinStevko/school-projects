%%%%%%%%%%%%%%%%%%%%%%%%%%%%%%% Riesenie %%%%%%%%%%%%%%%%%%%%%%%%%%%%%%%

Najprv dokážem pomocné tvrdenie. Majme $3^k$ gulí, pričom $3^k=2n+1$, 
ktoré máme rozdelené na 2 skupiny početné $n+1$ (označme $P$) a $n$ 
(označme $Q$), pričom vieme, že gule skupiny $P$ sú buď všetky pravé, 
alebo 1 z nich je ľahšia a gule skupiny $Q$ sú buď všetky pravé, alebo 
je 1 z nich ťažšia. Potom nepravú guľu vieme nájsť na $k$ vážení. 
Rovnaké tvrdenie budeme uvažovať aj pre opčný smer, teda že ľahšia je 
v $P$, alebo ťažšia v $Q$, ale dokážem len tento smer a druhý sa ukáže 
analogicky. Nech teda $g_1, g_2,\dots, g_{n+1}\in P$ a 
$g_{n+2}, g_{n+3},\dots, g_{2n+1}\in Q$.

Dokážem indukciou. Pre $k=1$ máme 3 gule. Porovnám gule $g_1$ a $g_2$. 
Ak sú rovnaké, guľa $g_3$ je ťažšia. Ak nie sú rovnaké viem, že medzi 
nimi môže byť len ľahšia guľa a teda guľa na tej strane, ktorá je 
vyššie je ľahšia. 

Nech teda tvrdenie vyššie platí pre $k$ a dokážeme indukčný krok pre 
$k+1$. Podľa $n$ značenia máme v predpoklade $2n+1$ gulí a v indukčnom 
kroku $6n+3$ gulí. Bez ujmy na všeobecnosti si teda z tvrdenia vieme 
povedať, že gule $g_1, g_2,\dots, g_{3n+2}\in P$ a gule 
$g_{3n+3}, g_{3n+4},\dots, g_{6n+3}\in Q$. Rozdeľme si tieto gule na 
3 skupiny. 
\begin{itemize}
    \item $g_1, g_2,\dots, g_{n+1}, 
    g_{3n+3}, g_{3n+4},\dots, g_{4n+2}$
    \item $g_{n+2}, g_{n+3},\dots, g_{2n+2}, 
    g_{4n+3}, g_{4n+3},\dots, g_{5n+2}$
    \item $g_{2n+3}, g_{2n+4},\dots, g_{3n+2}, 
    g_{5n+3}, g_{5n+4},\dots, g_{6n+3}$
\end{itemize}
Teraz sú všetky skupiny rovnako, konkrétne $2n+1$ početné. Porovnajme 
prvé 2 skupiny. Ak sú rovnaké, môžu obsahovať len pravé gule a potom 
v tretej skupine musí byť nepravá. V nej nám ale ostali gule zadelené 
do $P$ a $Q$ podľa podmienok tvrdenia, takže môžeme využiť indukčný 
predpoklad. Ak je jedna strana ťažšia viem, že buď je na ťažšej strane 
ťažšia guľa, alebo na ľahšej ľahšia. Potom ale viem vybrať gule z $P$ 
z ľahšej strany a gule z $Q$ z tej ťažšej a povedať, že nepravá guľa 
je medzi vybranými. Na tieto vybrané potom môžem aplikovať indukčný 
predpoklad, keďže spĺňajú podmienky zadelenia v tvrdení. Avšak v oboch 
prípadoch som 1 váženie spotreboval a ostalo mi ich $k$, pričom na 
indukčný predpoklad potrebujem práve $k$ vážení. Z toho ale vyplýva, že
každých $3^k$ gulí teda odvážim na $k$ vážení, čo som chcel dokázať.

Pozrime sa teraz na samotnú úlohu. Máme teda $n=\frac{3^k-1}{2}$ gulí 
a chceme dokázať, že ich dokážeme odvážiť na $k$ vážení. Znova 
indukciou pre $k=1$ máme 1 guľu, tú odvážime so správnou a vieme 
aká je. Predpokladajme teda, že tvrdenie platí pre $k$ a dokážme ho 
pre $k+1$. Z našich $\frac{3^{k+1}-1}{2}$ si teraz odčlenme 
$\frac{3^k-1}{2}$ gulí, pretože podľa indukčného predpokladu vieme, 
že ak si budeme istí, že nepravá guľa je medzi nimi, ktorá to je 
dokážeme zistiť na $k$ vážení. Ostalo nám teda 
\[\dfrac{3^{k+1}-1}{2}-\dfrac{3^k-1}{2}=3^k\]
gulí. Všimnime si teraz ale, že ak výstup z ich váženia (pričom pri 
vážení ich rozdelíme na polovicu a k menšej časti pridáme správnu 
guľu, ktorú po vážení odoberieme) použijeme ako vstup pre vyššie 
dokázaný pomocný argument, tak to, ktorá z nich je nepravá zistíme 
na $k$ vážení. 1 ďalšie sme potrebovali na ich zadelenie a teda 
z $\frac{3^{k+1}-1}{2}$ gulí vieme identifikovať nepravú na $k+1$ 
vážení. Podľa dôkazu z cvičenia je to zároveň maximum, takže toto 
nemusíme ďalej dokazovať. 

V dôkaze je zároveň popísaný algoritmus váženia, ku ktorému môžeme 
pristupovať rekurzívne, až kým neidentifikujeme nesprávnu guľu. 
V prípade, že gulí nemáme maximálny možný počet na $k$ vážení, ale 
menší, ako z algoritmu vyplýva, dôležité je len ich rozdelenie, 
početnosti môžu byť aj menšie.

%%%%%%%%%%%%%%%%%%%%%%%%%%%%%%%%%%%%%%%%%%%%%%%%%%%%%%%%%%%%%%%%%%%%%%%%
