%%%%%%%%%%%%%%%%%%%%%%%%%%%%%%% Riesenie %%%%%%%%%%%%%%%%%%%%%%%%%%%%%%%

Keďže pán Novák (budem značiť mrN) dostal od každého inú odpoveď a 
mohol dostať iba odpovede 0, 1, 2, 3, 4, 5, 6 a pýtal sa siedmich ľudí, 
každú odpoveď dostal práve raz.

Pani Nováková (budem značiť msN) si nemohla podať ruku so 6 ľudmi, lebo 
inak by iba mrN mohol odpovedať 0, no sám seba sa nepýtal. Zoberme si 
teda pár msA a mrA, kde si mrA podal ruku so 6 ľuďmi. Potom msA je 
posledná, ktorá si nepodala ruku s nikým a teda ona musela odpovedať 0. 

Odoberme teraz pár A. Ostávajú 3 páry a odpovede 1, 2, 3, 4, 5. Keďže 
ale doposiaľ si každý z nich podal ruku práve s 1, pre jednoduchosť 
zmeňme odpovede na o 1 nižšie a toto podanie ignorujme. Máme teda 
odpovede 0, 1, 2, 3, 4. 

Pani Nováková si nemohla podať ruku so 4 ľudmi, lebo inak by iba mrN 
mohol odpovedať 0, no sám seba sa nepýtal. Zoberme si teda pár msB a 
mrB, kde si mrB podal ruku so 4 ľuďmi. Potom msB je posledná, ktorá si 
nepodala ruku s nikým a teda ona musela odpovedať 0. 

Odoberme teraz pár B. Ostávajú 2 páry a odpovede (pôvodné) 2, 3, 4. 
Keďže ale doposiaľ si každý z nich podal ruku práve s 2, pre 
jednoduchosť zmeňme odpovede na o 2 nižšie a tieto podania ignorujme. 
Máme teda odpovede 0, 1, 2. 

Pani Nováková si nemohla podať ruku s 2 ľudmi, lebo inak by iba mrN 
mohol odpovedať 0, no sám seba sa nepýtal. Zoberme si teda pár msC a 
mrC, kde si mrC podal ruku s 2 ľuďmi. Potom msC je posledná, ktorá si 
nepodala ruku s nikým a teda ona musela odpovedať 0. 

Odoberme teraz pár C. Ostáva iba pár N a odpoveď (pôvodná) 3. MsN si 
teda musela podať ruku s práve 3 ľuďmi. 

%%%%%%%%%%%%%%%%%%%%%%%%%%%%%%%%%%%%%%%%%%%%%%%%%%%%%%%%%%%%%%%%%%%%%%%%
