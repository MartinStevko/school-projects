%%%%%%%%%%%%%%%%%%%%%%%%%%%%%%% Riesenie %%%%%%%%%%%%%%%%%%%%%%%%%%%%%%%

Dokážem pomocné tvrdenie, že v grafe existuje list, alebo kružnica. 
Ak tam neexistuje list, potom každý vrchol má stupeň aspoň 2 a teda 
ak idem len po hranách, po ktorých som ešte nešiel, buď môžem ísť 
ďalej, alebo narazím na vrchol, z ktorého vedú iba hrany, v ktorých 
som už bol. Potom som ale našiel kružnicu. Ak tam neexistuje 
kružnica a znova idem len po hranách, kde som ešte nebol, buď môžem 
ísť ďalej, alebo narazím na list, keďže vo vrchole kam prídem nemôže 
existovať žiadna hrana, kde som už bol (to by som našiel kružnicu). 

Naviac viem povedať, že ak v grafe neexistuje kružnica, existujú tam 
aspoň 2 listy. Ak by totiž existoval iba jeden, zažnem v ňom a 
zopakujem rovnakú úvahu ako vyššie, čím dôjdem k ďalšiemu. Ak existujú 
aspoň 2 listy, viem ich odobrať a úlohu mám dokázanú. 

Ak nie, mám tam 
kružnicu. Vyberiem teda jeden bod kružnice a znova tam mám buď list, 
alebo kružnicu. Potom vyberiem aj tento list/bod kružnice. Viem, že 
graf ostal súvislý po odobraní 2 vrcholov, bol súvislý po odobraní 
aj iba prvého a bude súvislý aj po odobraní iba druhého, lebo pridám 
aspoň 2 hrany, ktoré nemôžu ísť do toho istého vrcholu.

%%%%%%%%%%%%%%%%%%%%%%%%%%%%%%%%%%%%%%%%%%%%%%%%%%%%%%%%%%%%%%%%%%%%%%%%
