%%%%%%%%%%%%%%%%%%%%%%%%%%%%%%% Riesenie %%%%%%%%%%%%%%%%%%%%%%%%%%%%%%%

Začnem s výpočtom inverznej matice v $\z_3$.
\[\begin{pmatrix}[cccc|cccc]
    1 & 0 & 1 & 1 & 1 & 0 & 0 & 0 \\
    2 & 0 & 1 & 1 & 0 & 1 & 0 & 0 \\
    2 & 1 & 0 & 0 & 0 & 0 & 1 & 0 \\
    1 & 2 & 1 & 0 & 0 & 0 & 0 & 1 \\
   \end{pmatrix}\]
Pričítam 1. riadok k 2. a 3. raz a ku 4. dvakrát.
\[\begin{pmatrix}[cccc|cccc]
    1 & 0 & 1 & 1 & 1 & 0 & 0 & 0 \\
    0 & 0 & 2 & 2 & 1 & 1 & 0 & 0 \\
    0 & 1 & 1 & 1 & 1 & 0 & 1 & 0 \\
    0 & 2 & 0 & 2 & 2 & 0 & 0 & 1 \\
   \end{pmatrix}\]
3. riadok dám na 2. miesto, 2. na 4. miesto a ku pôvodnému 4. 
pripočítam pôvodný 3. a dám ho na 3. miesto.
\[\begin{pmatrix}[cccc|cccc]
    1 & 0 & 1 & 1 & 1 & 0 & 0 & 0 \\
    0 & 1 & 1 & 1 & 1 & 0 & 1 & 0 \\
    0 & 0 & 1 & 0 & 0 & 0 & 1 & 1 \\
    0 & 0 & 2 & 2 & 1 & 1 & 0 & 0 \\
   \end{pmatrix}\]
4. riadok najprv pričítam k 1. a 2. a potom k nemu pričítam 3. riadok.
\[\begin{pmatrix}[cccc|cccc]
    1 & 0 & 0 & 0 & 2 & 1 & 0 & 0 \\
    0 & 1 & 0 & 0 & 2 & 1 & 1 & 0 \\
    0 & 0 & 1 & 0 & 0 & 0 & 1 & 1 \\
    0 & 0 & 0 & 2 & 1 & 1 & 1 & 1 \\
   \end{pmatrix}\]
4. riadok vynásobim dvomi.
\[\begin{pmatrix}[cccc|cccc]
    1 & 0 & 0 & 0 & 2 & 1 & 0 & 0 \\
    0 & 1 & 0 & 0 & 2 & 1 & 1 & 0 \\
    0 & 0 & 1 & 0 & 0 & 0 & 1 & 1 \\
    0 & 0 & 0 & 1 & 2 & 2 & 2 & 2 \\
   \end{pmatrix}\]
Na ľavej strane sme dostali jednotkovú maticu, takže tá na strane 
pravej bude inverzná k matici v zadaní. Pre skúšku môžeme matice 
v $\z_3$ vynásobiť a skutočne dostaneme jednotkovú, takže sú inverzné.

\bigskip
Teraz vypočítam inverznú maticu v $\z_5$.
\[\begin{pmatrix}[cccc|cccc]
    1 & 0 & 1 & 1 & 1 & 0 & 0 & 0 \\
    2 & 0 & 1 & 1 & 0 & 1 & 0 & 0 \\
    2 & 1 & 0 & 0 & 0 & 0 & 1 & 0 \\
    1 & 2 & 1 & 0 & 0 & 0 & 0 & 1 \\
   \end{pmatrix}\]
1. riadok odpočítam od 2. a 3. dvakrát a od 4. raz.
\[\begin{pmatrix}[cccc|cccc]
    1 & 0 & 1 & 1 & 1 & 0 & 0 & 0 \\
    0 & 0 & 4 & 4 & 3 & 1 & 0 & 0 \\
    0 & 1 & 3 & 3 & 3 & 0 & 1 & 0 \\
    0 & 2 & 0 & 4 & 4 & 0 & 0 & 1 \\
   \end{pmatrix}\]
Najprv k 1. riadku pripočítam 2. a potom dám 3. riadok na 2. miesto, 
2. na 4. miesto a 4. na 3. miesto.
\[\begin{pmatrix}[cccc|cccc]
    1 & 0 & 0 & 0 & 4 & 1 & 0 & 0 \\
    0 & 1 & 3 & 3 & 3 & 0 & 1 & 0 \\
    0 & 2 & 0 & 4 & 4 & 0 & 0 & 1 \\
    0 & 0 & 4 & 4 & 3 & 1 & 0 & 0 \\
   \end{pmatrix}\]
Od 3. riadku dvakrát odpočítam 2. a 4. vynásobím štyroma.
\[\begin{pmatrix}[cccc|cccc]
    1 & 0 & 0 & 0 & 4 & 1 & 0 & 0 \\
    0 & 1 & 3 & 3 & 3 & 0 & 1 & 0 \\
    0 & 0 & 4 & 3 & 3 & 0 & 3 & 1 \\
    0 & 0 & 1 & 1 & 2 & 4 & 0 & 0 \\
   \end{pmatrix}\]
K 2. a 3. riadku pripočítam dvojnásobok 4. riadku a následne od 4. 
riadku odpočítam nový 3. riadok.
\[\begin{pmatrix}[cccc|cccc]
    1 & 0 & 0 & 0 & 4 & 1 & 0 & 0 \\
    0 & 1 & 0 & 0 & 2 & 3 & 1 & 0 \\
    0 & 0 & 1 & 0 & 2 & 3 & 3 & 1 \\
    0 & 0 & 0 & 1 & 0 & 1 & 2 & 4 \\
   \end{pmatrix}\]
Na ľavej strane sme znova dostali jednotkovú maticu, takže tá na strane 
pravej bude inverzná k matici v zadaní. Pre skúšku môžeme matice 
v $\z_5$ vynásobiť a skutočne dostaneme jednotkovú, takže sú inverzné.

%%%%%%%%%%%%%%%%%%%%%%%%%%%%%%%%%%%%%%%%%%%%%%%%%%%%%%%%%%%%%%%%%%%%%%%%
