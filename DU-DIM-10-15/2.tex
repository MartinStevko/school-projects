%%%%%%%%%%%%%%%%%%%%%%%%%%%%%%% Riesenie %%%%%%%%%%%%%%%%%%%%%%%%%%%%%%%

\begin{itemize}
    \item $R\circ S$

    Zoberme si $X=\{1,2,3\}$, pričom $R=\{(1,1), (2,2), (3,3), 
    (1,3), (3,1)\}$ a $S=\{(1,1), (2,2), (3,3),\allowbreak (2,3), (3,2)\}$. 
    Obe tieto relácie sú ekvivalencie, pretože sú reflexívne, 
    symetrické a ku tranzitivite na nich nedochádza, takže aj 
    tranzitívne. Pre ich zloženie však platí $1(R\circ S)2$, pretože 
    $1R3$ a $3S2$ a neplatí $2(R\circ S)1$, pretože 2 je v $R$ 
    relácii iba s 2 a 2 je v $S$ relácii iba s 2 a 3, nie 1. Potom 
    relácia $(R\circ S)$ nie je symetrická a teda viem povedať, že 
    pre ňu nemusí platiť, že je ekvivalenciou. (Neplatí to napríklad 
    v prípade popísanom vyššie.)
    \item $R\cap S$

    To, že relácia je ekvivalencia dokážem ak ukážem, že je 
    reflexívna (využitím reflexivity $R$ a $S$)
    \[\forall x\in X: (x,x)\in R\wedge (x,x)\in S\Rightarrow 
    (x,x)\in (R\cap S)\]
    symetrická (využitím symetri $R$ a $S$)
    \begin{align*}
        x(R\cap S)y&\Leftrightarrow (x,y)\in (R\cap S)\\
        &\Leftrightarrow (x,y)\in R\wedge (x,y)\in S\\
        &\Leftrightarrow (y,x)\in R\wedge (y,x)\in S\\
        &\Leftrightarrow (y,x)\in (R\cap S)\\
        &\Leftrightarrow y(R\cap S)x
    \end{align*}
    aj tranzitívna (využitím tranzitivity $R$ a $S$)
    \begin{align*}
        x(R\cap S)y\wedge y(R\cap S)z&\Leftrightarrow 
        (x,y)\in (R\cap S)\wedge (y,z)\in (R\cap S)\\
        &\Leftrightarrow (x,y)\in R\wedge (x,y)\in S\wedge
        (y,z)\in R\wedge (y,z)\in S\\
        &\Rightarrow (x,z)\in R\wedge (x,z)\in S\\
        &\Leftrightarrow (x,z)\in (R\cap S)\\
        &\Leftrightarrow x(R\cap S)z
    \end{align*}
    Keďže $(R\cap S)$ je reflexívna, symetrická aj tranzitívna, je 
    to aj ekvivalencia.
    \item $R\cup S$

    Zoberme si $X=\{1,2,3\}$, pričom $R=\{(1,1), (2,2), (3,3), 
    (1,2), (2,1)\}$ a $S=\{(1,1), (2,2), (3,3),\allowbreak (2,3), (3,2)\}$. 
    Obe relácie sú zjavne reflexívne aj symetrické a ku tranzitivite 
    na nich nedochádza, takže aj tranzitívne. V relácii $(R\cup S)$ 
    ale ku tranzitivite už dochádza a neplatí tam, pretože 
    $(1,2), (2,3)\in (R\cup S)$ a zároveň $(1,3)\notin (R\cup S)$.
    Relácia $(R\cup S)$ teda nemusí byť ekvivalencia.
\end{itemize}

%%%%%%%%%%%%%%%%%%%%%%%%%%%%%%%%%%%%%%%%%%%%%%%%%%%%%%%%%%%%%%%%%%%%%%%%
