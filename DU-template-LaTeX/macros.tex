%%%%%%%%%%%%%%%%%%%%%%%%% Dokument %%%%%%%%%%%%%%%%%%%%%%%%%
\usepackage[top=2.5cm, bottom=2.5cm, left=2cm, right=2cm]{geometry}

\usepackage[utf8]{inputenc}
\usepackage[IL2]{fontenc}
%\usepackage[slovak]{babel}
%\usepackage{a4wide}
\usepackage{amsmath,amsfonts,amssymb}
\usepackage{comment}
\usepackage{enumitem}

\usepackage{graphicx}   % obrazky
\usepackage{wrapfig}    % obrazky

\usepackage{caption}    % popisky
\usepackage{subcaption} % popisky

\usepackage{listings}   % python
\usepackage{color}      % python

\usepackage{array}      % column width
\usepackage{calc}       % pocitanie vo funkciach (vspace,...)

\usepackage[shellescape,latex]{gmp}     % metapost
\usepackage{hyperref}   % url

\usepackage{siunitx}    % stvorcek (QED)
\setlength{\fboxsep}{.5\fboxsep}

\usepackage{pgfplots}   % grafy
\usepackage{pgf}        % grafy
\pgfplotsset{compat=1.15}

\usepackage{tikz}       % kreslenie

\pagenumbering{gobble}
%%%%%%%%%%%%%%%%%%%%%%%%%%%%%%%%%%%%%%%%%%%%%%%%%%%%%%%%%%%%

%%%%%%%%%%%%%%%%%%%%%%%%% Hlavicka a riesenie %%%%%%%%%%%%%%%%%%%%%%%%%
\newcount\uloha
\newtoks\meno
\newtoks\skola

\parindent=0pt
\parskip=5pt

\def\hlavicka{%
    \the\meno \par
    \the\skola \par
    \underline{Úloha \the\uloha.} \par
}

\def\riesenie{%
\parindent=20pt
\parskip=5pt}
%%%%%%%%%%%%%%%%%%%%%%%%%%%%%%%%%%%%%%%%%%%%%%%%%%%%%%%%%%%%

%%%%%%%%%%%%%%%%%%%%%%%%% Maticke makra %%%%%%%%%%%%%%%%%%%%%%%%%
\def\n{{\mathbb N}}
\def\z{{\mathbb Z}}
\def\q{{\mathbb Q}}
\def\r{{\mathbb R}}

\def\vus#1{|{#1}|}
\def\uh{\sphericalangle}
\def\vu#1{|\sphericalangle #1|}
\def\st{{}^\circ}

\def\priamka#1{\overleftrightarrow{#1}}
\def\vektor#1{\overrightarrow{(#1)}}

\def\kolme{\perp}
\def\rovnobezne{\mathop{\|}}
\def\zhodny{\cong}
\def\podobny{\sim}

\def\obsah#1{{\mathrm S}_{#1}}
\def\vzdialenost#1#2{\vus{{#1},{#2}}}

\def\kongruentne{\equiv}
\def\kong#1#2#3{$#1\equiv#2\pmod{#3}$}
\def\deli{\mathop{|}}
\def\nedeli{\mathop{\not|}}
\def\dcc#1{\left\lfloor{#1}\right\rfloor}
\def\hcc#1{\left\lceil{#1}\right\rceil}
\def\alebo{\vee}
\def\zaroven{\wedge}

\def\mm{\mathrm{mm}}
\def\cm{\mathrm{cm}}
\def\dm{\mathrm{dm}}
\def\m{\mathrm{m}}
\def\km{\mathrm{km}}
\def\kg{\mathrm{kg}}

\makeatletter
\renewcommand*\env@matrix[1][*\c@MaxMatrixCols c]{%
   \hskip -\arraycolsep
   \let\@ifnextchar\new@ifnextchar
   \array{#1}}
\makeatother
%%%%%%%%%%%%%%%%%%%%%%%%%%%%%%%%%%%%%%%%%%%%%%%%%%%%%%%%%%%%

%%%%%%%%%%%%%%%%%%%%%%%%% Ine makra %%%%%%%%%%%%%%%%%%%%%%%%%
\def\python#1{\lstinputlisting[language=Python]{#1}}

\def\obr#1#2{%
\begin{center}
\includegraphics[width=#1\textwidth]{#2}
\end{center}
}

\def\wrapobr#1#2#3#4#5#6{%
\begin{wrapfigure}[#1]{#2}{#3\textwidth}
\vspace{#4pt}
\begin{center}
\includegraphics[width=#5\textwidth]{images/#6}
\end{center}
\end{wrapfigure}
}

\def\wrapmpost#1#2#3#4#5#6{%
\begin{wrapfigure}[#1]{#2}{#3\textwidth}
\vspace{#4pt}
\centering
\resizebox{#5\textwidth}{!}{\input{images/#6}}
\end{wrapfigure}
}

\newcolumntype{C}[1]{>{\centering\let\newline\\\arraybackslash\hspace{0pt}}m{#1}}
\newcolumntype{L}[1]{>{\leftflushing\let\newline\\\arraybackslash\hspace{0pt}}m{#1}}
\newcolumntype{R}[1]{>{\rightflushing\let\newline\\\arraybackslash\hspace{0pt}}m{#1}}

\def\stvorcek#1{%
\hspace{#1}\fbox{$\phantom{5}$}
}
%%%%%%%%%%%%%%%%%%%%%%%%%%%%%%%%%%%%%%%%%%%%%%%%%%%%%%%%%%%%

%%%%%%%%%%%%%%%%%%%%%%%%%%% Toks %%%%%%%%%%%%%%%%%%%%%%%%%%%
\meno={Martin Števko}
\skola={MFF UK, Informatika, 1. ročník}
%%%%%%%%%%%%%%%%%%%%%%%%%%%%%%%%%%%%%%%%%%%%%%%%%%%%%%%%%%%%

%%%%%%%%%%%%%%%%%%%%%%%%% Zoznam prikazov %%%%%%%%%%%%%%%%%%%%%%%%%
\begin{comment}
\obr{pocet obtekanych riadkov}{l/r}{sirka}{cesta\nazov}

 ~                              nedelitelna medzera
 \                              medzera v matickom texte
 \\                             prechod na novy riadok
 \uv{}                          uvodzovky
 \textit{}                      sikmy text
 \textbf{}                      tucny text
 \underline{}                   podciarknuty text
 \overline{}                    nadciarknuty text
 ^                              horny index
 _                              dolny index
 \sqrt[n]{k}                    n-ta odmocnina z k
 \dfrac{citatel}{menovatel}     zlomok
 \begin{align}
     ...&=...\\
 \end{align}                    sustavy rovnic so symbolom & pred                                           rovna sa zarovna
 \nonumber                      pre necislovane align
 \begin{align*}
     ...&=...\\
 \end{align*}                   riadky necisluje
 \rightarrow                    sipka vpravo
 \leftarrow                     sipka vlavo
 \leftrightarrow                obojsmerna sipka (ekvivalencia)
 \in                            mnozinovy znak "patri"
 \colon                         dvojbodka
 \dbinom{hore}{dole}            binomicke cislo (kombinacne)
 \cos, \sin, \tan               goniometricke funkcie
 \alpha, \beta,...              grecka abeceda
 \pm                            znamienko plus-minus
 \log                           logaritmus
 \displaystyle\sum_{}^{}        suma
 \displaystyle\prof_{}^{}       sucin
 \displaystyle\lim_{x\to0}      limita do 0
 \infty                         nekonecno
 \begin{itemize}
	\item 
 \end{itemize}                  odrazky
 \begin{tabular}{|c|c|p{2cm}|c|}
    \hline
     & $A$ & $B$ \\
    \hline
    $1.$ & $0$ - $9$ & $0$ - $9$\\
    \hline
 \end{tabular}                  tabulka, & oddeluje stlpce, \\riadky,                                       \hline kresli ciari, (l,c,r) zarovnava,                                     p{2cm} robi medzeru medzi stlpcami
\python                         import python kodu

%%%%%%%%%%%%%%%%%%%%%%%%%%%%%%%%%%%%%%%%%%%%%%%%%%%%%%%%%%%%

%%%%%%%%%%%%%%%%%%%%%%%%% Input obrazku %%%%%%%%%%%%%%%%%%%%%%%%%
\begin{figure}[h]
\begin{flushleft/center/flushright}
\includegraphics[scale=1]{cesta\nazov}
%\includegraphics[scale=1]{cesta\nazov}
%\includegraphics[scale=1]{cesta\nazov}    %%% (viac do riadku)
%\caption{popisok pod}
\end{flushleft/center/flushright}
\end{figure}                                vlozenie obrazku

\begin{figure}
    \centering
    \begin{subfigure}[b]{0.3\textwidth}
        \includegraphics[width=\textwidth]{gull}
        \caption{A gull}
        \label{fig:gull}
    \end{subfigure}
    ~ %add desired spacing between images, e. g. ~, \quad, \qquad, \hfill etc. 
      %(or a blank line to force the subfigure onto a new line)
    \begin{subfigure}[b]{0.3\textwidth}
        \includegraphics[width=\textwidth]{tiger}
        \caption{A tiger}
        \label{fig:tiger}
    \end{subfigure}
    ~ %add desired spacing between images, e. g. ~, \quad, \qquad, \hfill etc. 
    %(or a blank line to force the subfigure onto a new line)
    \begin{subfigure}[b]{0.3\textwidth}
        \includegraphics[width=\textwidth]{mouse}
        \caption{A mouse}
        \label{fig:mouse}
    \end{subfigure}
    \caption{Pictures of animals}\label{fig:animals}
\end{figure}

\begin{wrapfigure}[pocet obtekanych riadkov]{l/r}{0.52\textwidth}
\begin{center}
\includegraphics[width=0.5\textwidth]{cesta\nazov}
\end{center}
%\caption{popisok pod}
\end{wrapfigure}                            s obtekanim textu
%%%%%%%%%%%%%%%%%%%%%%%%%%%%%%%%%%%%%%%%%%%%%%%%%%%%%%%%%%%%

%%%%%%%%%%%%%%%%%%%%%%%%% Pocitadla %%%%%%%%%%%%%%%%%%%%%%%%%
\setcounter{page}{1}                nastavenie predvoleneho pocitadla stran
\newcount\pocet                     definovanie pocitadla pocet
\pocet = 1                          priradenie hodnoty pocitadlu
\the\pocet                          pouzitie pocitadla pocet
\global\advance\pocet by 1          zvysenie pocitadla o 1
\addtocounter{pocet}{cislo}         do pocet prida cislo
\end{comment}
%%%%%%%%%%%%%%%%%%%%%%%%%%%%%%%%%%%%%%%%%%%%%%%%%%%%%%%%%%%%