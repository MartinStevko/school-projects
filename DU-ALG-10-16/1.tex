%%%%%%%%%%%%%%%%%%%%%%%%%%%%%%% Riesenie %%%%%%%%%%%%%%%%%%%%%%%%%%%%%%%

Naprogramovaný algoritmus je v súbore $u1.py$.

Algoritmom, ktorý hľadá stabilné párovanie je napríklad tento:
\begin{itemize}
    \item v každom kole každý muž, ktorého si žena "neodložila" pošle 
    návrh žene, ktorej ešte návrh neposielal
    \item každá žena, ktorej prišli nejaké návrhy si nechá najlepší 
    z nich a zvyšné odmietne
    \item po najviac $n$ kolách algoritmus skončí, pričom každá žena 
    bude mať priradeného práve 1 muža a toto párovanie bude stabilné.
\end{itemize}

Teraz musím dokázať 3 veci -- algoritmus skončí, skončí po najviac 
$n^2-n+1$ krokoch a párovanie, ktoré vznikne bude stabilné. 

Keďže v každom kole je po podaní návrhov $n$ neodmietnutých návrhov, 
existujú 2 možnosti. Buď má každá žena návrh od práve 1 muža a vtedy 
algoritmus skončí, alebo má nejaké žena viacero návrhov. Ak nastala 
druhá situácia, táto žena všetky okrem jedného zo svojich návrhov 
odmietne, čím sa počet neodmietnutých návrhov zníži. Keďže počet 
návrhov je konečný a v každom kole kedy sa algoritmus neskončí 
sa ich počet zníži, algoritmus musí skončiť, najneskôr vtedy, keď 
ostane práve $n$ zo všetkých neodmietnutých návrhov.

Po4et možných návrhov na podanie je $n^2$. V prvom kole sa podá $n$ 
návrhov. Ak by sa v každom ďalšom kole podal iba $1$ návrh, ako 
som dokázal vyššie, návrhy sa vyčerpajú najviac po $n^2-n+1$ kolách 
a teda najneskôr vtedy skončí aj celý algoritmus.

Predstavme si situáciu v ktorej algoritmus skončí. Každý muž má podľa 
jeho osobných preferencií najlepšiu ženu, ktorá ho neodmietla. 
Žiaden muž teda pár chcieť meniť nebude (lebo všetky podľa neho lepšie 
ženy ho odmietnu) a teda každý pár, čiže aj celé párovanie je stabilné. 

%%%%%%%%%%%%%%%%%%%%%%%%%%%%%%%%%%%%%%%%%%%%%%%%%%%%%%%%%%%%%%%%%%%%%%%%
