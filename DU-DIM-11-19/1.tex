%%%%%%%%%%%%%%%%%%%%%%%%%%%%%%% Riesenie %%%%%%%%%%%%%%%%%%%%%%%%%%%%%%%

Pre spor predpokladajme, že najdlhšie cesty (označme $a$ a $b$) nemajú 
žiaden spoločný bod. Potom na to, aby bol graf súvislý musí existovať 
cesta $c$ vedúca z $a$ do $b$ (aj naopak, keďže hovoríme 
o neorientovaných grafoch). Zoberme si teda dlhšiu z dvoch častí cesty 
$a$ rozdelenej cestou $c$. Tá má dĺžku aspoň $a/2$. Taktiež si zoberme 
dlhšiu z dvoch častí cesty $b$ rozdelenej cestou $c$. Tá má tiež dĺžku 
aspoň $b/2=a/2$, keďže obe cesty sú najdlhšie a teda aj rovnako dlhé. 
Nakoniec dĺžka cesty $c$ je aspoň 1 a potom prepojením vybraných častí 
cestou $c$ dostávame cestu dĺžky aspoň $a+$, čo je spor s tým, že cesty 
$a$ a $b$ boli najdlhšie. Tieto teda musia mať aspoň 1 spoločný bod.

%%%%%%%%%%%%%%%%%%%%%%%%%%%%%%%%%%%%%%%%%%%%%%%%%%%%%%%%%%%%%%%%%%%%%%%%
