%%%%%%%%%%%%%%%%%%%%%%%%%%%%%%% Riesenie %%%%%%%%%%%%%%%%%%%%%%%%%%%%%%%

Dokážem, že $R\circ R\Rightarrow R$ a naopak. Z toho potom nutne 
vyplýva ekvivalencia, teda aj rovnosť.

Podľa definície zloženia relácií platí
\[\forall x,z: x(R\circ R)z\Leftrightarrow\forall x,z \exists y\in X: 
xRy\wedge yRz\]
potom ale z tranzitivity
\[\forall x,z\exists y\in X: xRy\wedge yRz\Rightarrow \forall x,z xRz\]

Pre opačný smer zo symetrie a tranzitivity platí, že ak
\[xRz\wedge\exists y\in X, x\neq y: xRy\Rightarrow yRz\Leftrightarrow 
x(R\circ R)z\]
Ak také $y$ neexistuje pre nejakú dvojicu $x, z$, potom nutne je táto 
dvojica v relácii iba v rámci seba, teda so žiadným iným prvkom (kvôli 
symetrii a tranzitivite). Z toho ale vyplýva, že pre všetky takéto 
$x, z$ môžem zobrať $y=x$ a potom platí 
\[xRx\wedge xRz\Rightarrow x(R\circ R)z\]

Keďže platí obojstranná implikácia (pre všetky prípady), platí aj 
ekvivalencia, teda $R\circ R=R$, čo som mal dokázať.

%%%%%%%%%%%%%%%%%%%%%%%%%%%%%%%%%%%%%%%%%%%%%%%%%%%%%%%%%%%%%%%%%%%%%%%%
