%%%%%%%%%%%%%%%%%%%%%%%%%%%%%%% Riesenie %%%%%%%%%%%%%%%%%%%%%%%%%%%%%%%

\begin{enumerate}
    \item Matica prechodu od bázy $B$ do kanoickej vyzerá ako 
    $${}_{K}[id]_{B}=
    \begin{pmatrix}
        1 & 0 & 2 \\
        1 & 1 & 0 \\
        1 & -1 & 1 \\
    \end{pmatrix}$$
    \item Matica prechodu od bázy $B'$ do kanoickej vyzerá ako
    $${}_{K}[id]_{B'}=
    \begin{pmatrix}
        3 & 1 & 1 \\
        2 & 0 & 2 \\
        2 & 1 & 2 \\
    \end{pmatrix}$$
    avšak my chceme maticu opačného prechodu, takže musím nájsť jej inverz
    \[\begin{pmatrix}[ccc|ccc]
        3 & 1 & 1 & 1 & 0 & 0 \\
        2 & 0 & 2 & 0 & 1 & 0 \\
        2 & 1 & 2 & 0 & 0 & 1 \\
      \end{pmatrix}
      \sim
      \begin{pmatrix}[ccc|ccc]
        6 & 2 & 2 & 2 & 0 & 0 \\
        2 & 0 & 2 & 0 & 1 & 0 \\
        0 & 1 & 0 & 0 & -1 & 1 \\
      \end{pmatrix}
      \sim
      \begin{pmatrix}[ccc|ccc]
        0 & 2 & -4 & 2 & -3 & 0 \\
        2 & 0 & 2 & 0 & 1 & 0 \\
        0 & 1 & 0 & 0 & -1 & 1 \\
      \end{pmatrix}
      \sim\]
    \[\sim
      \begin{pmatrix}[ccc|ccc]
        2 & 0 & 2 & 0 & 1 & 0 \\
        0 & 1 & 0 & 0 & -1 & 1 \\
        0 & 2 & -4 & 2 & -3 & 0 \\
      \end{pmatrix}
      \sim
      \begin{pmatrix}[ccc|ccc]
        -2 & 0 & 2 & 0 & 1 & 0 \\
        0 & 1 & 0 & 0 & -1 & 1 \\
        0 & 0 & -4 & 2 & -1 & -2 \\
      \end{pmatrix}
      \sim
      \begin{pmatrix}[ccc|ccc]
        1 & 0 & 0 & \frac{1}{2} & \frac{1}{4} & -\frac{1}{2} \\
        0 & 1 & 0 & 0 & -1 & 1 \\
        0 & 0 & 1 & -\frac{1}{2} & \frac{1}{4} & \frac{1}{2} \\
      \end{pmatrix}\]
    \item Matica prechodu od bázy $B$ do bázy $B'$ potom vyzerá ako
    \[\begin{pmatrix}
        \frac{1}{2} & \frac{1}{4} & -\frac{1}{2} \\
        0 & -1 & 1 \\
        -\frac{1}{2} & \frac{1}{4} & \frac{1}{2} \\
      \end{pmatrix}
      \begin{pmatrix}
        1 & 0 & 2 \\
        1 & 1 & 0 \\
        1 & -1 & 1 \\
      \end{pmatrix}
      =
      \begin{pmatrix}
        \frac{1}{4} & \frac{3}{4} & \frac{1}{2} \\
        0 & -2 & 1 \\
        \frac{1}{4} & -\frac{1}{4} & -\frac{1}{2} \\
      \end{pmatrix}\]
    \item Ak teda máme zistiť súradnice vektora v báze $B'$, pričom 
    poznáme tie v báze $B$, stačí nám ho zľava vynásobiť maticou 
    prechodu.
    \[\begin{pmatrix}
        \frac{1}{4} & \frac{3}{4} & \frac{1}{2} \\
        0 & -2 & 1 \\
        \frac{1}{4} & -\frac{1}{4} & -\frac{1}{2} \\
      \end{pmatrix}
      \begin{pmatrix}
        1 \\
        2 \\
        0 \\
      \end{pmatrix}
      =
      \begin{pmatrix}
        \frac{7}{4} \\
        -4 \\
        -\frac{1}{4} \\
      \end{pmatrix}\]
    Tieto súradnice v báze $B'$ reprezentujú vektor $(1, 3, -1)$, ktorý 
    odpovedá vektoru zo zadania, takže matica prechodu aj súradnice sú 
    správne.
\end{enumerate}

%%%%%%%%%%%%%%%%%%%%%%%%%%%%%%%%%%%%%%%%%%%%%%%%%%%%%%%%%%%%%%%%%%%%%%%%
