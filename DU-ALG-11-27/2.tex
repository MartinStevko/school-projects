%%%%%%%%%%%%%%%%%%%%%%%%%%%%%%% Riesenie %%%%%%%%%%%%%%%%%%%%%%%%%%%%%%%

Nezávislosť je naprogramovaná v súbore $nezavislost.py$ a dominancia zase 
v $dominancia.py$. Oba algoritmy by sa dali optimalizovať ešte viac, 
ak by som o vstupe vedel viacero údajov -- napríklad, že šachovnica bude 
štvorcová. Pre rozmery šachovnice $x$ a $y$, typ figúrky ľubovoľný okrem 
pešiaka a šachovnicu ako pneumatiku sa mi to podarilo optimalizovať 
najviac takto.

Využil som rekurzívne volanie funkcie na skúšanie všetkých 
perspektívnych možností, teda tých, ktoré ešte nie sú ohrozené žiadnou 
figúrkou. Pre každú figúrku som nastavil aj základný matematický 
invariant, ktorý určuje minimum/maximum, akurát nedokazuje, že je 
dosiahnuteľné a teda ak program nájde nejaké usporiadanie pre invariant, 
môže skončiť.

Využité invarianty:
\begin{itemize}
    \item v jednom riadku ani stĺpci nemôžu byť 2 veže ani kráľovné,
    \item na podpolíčku $2\times 2$ nemôžu byť 2 králi,
    \item každý kôň ohrozuje 8 políčok rozdielnej farby a každé políčko
    je ohrozované maximálne 8 koňmi
    \item najdlhšia uhlopriečka prechádza cez počet políčok rovný menšej
    strane šachovnice
    \item žiaden kráľ ani kôň neohrozí viac než 8 políčok
    \item žiadna veža neohrozí 2 riadky ani 2 stĺpce.
\end{itemize}

%%%%%%%%%%%%%%%%%%%%%%%%%%%%%%%%%%%%%%%%%%%%%%%%%%%%%%%%%%%%%%%%%%%%%%%%
