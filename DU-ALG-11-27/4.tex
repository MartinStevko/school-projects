%%%%%%%%%%%%%%%%%%%%%%%%%%%%%%% Riesenie %%%%%%%%%%%%%%%%%%%%%%%%%%%%%%%

Vytvorím si 2 polia, ktoré budem používať ako zásobníky. V jednom budem 
udržiavať nevyhodnotené znamienka a v druhom nevyhodnotené čísla. 

Vstup budem čítať po znakoch, ak dostanem znamienko pozriem sa čo je 
na vrchu zásobníku. Ak je to znamienko s vyššou alebo rovnakou prioritou, 
vyhodnotím najprv to a potom zopakujem pokus o vloženie, až kým ho 
nevložím. Ak prečítam zo vstupu ľavú zátvorku, dám ju ku znamienkam, ak 
pravú, nájdem jej ľavú zátvorku v zásobníku a vyhodnotím znamienka medzi 
nimi. Ak dostanem číslo, vložím ho do zásobníka. Následne, akonáhle 
prečítam celý vstup, vyhodnotím každé znamienko v zásobníku postupne.

Zátvorku budem považovať za najvyššiu prioritu.

Program je v súbore "infix.py". Ako znamienka môžu byť uvedené $+$, $-$, 
$*$ a $/$, pričom delenie je len celočíselné (no nie je problém rozšíriť 
na klasické, akurát by som musel celý čas rátať s typom float). Výraz 
môže takisto obsahovať ľubovoľné uzátvorkovanie, ale iba z okrúhlych 
zátvoriek $($, $)$ (pre jednoduchosť). Čísla môžu byť ľubovoľné kladné, 
záporné program na vstupe nezvládne, avšak rátať s nimi dokáže.

%%%%%%%%%%%%%%%%%%%%%%%%%%%%%%%%%%%%%%%%%%%%%%%%%%%%%%%%%%%%%%%%%%%%%%%%
