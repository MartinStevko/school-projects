%%%%%%%%%%%%%%%%%%%%%%%%%%%%%%% Riesenie %%%%%%%%%%%%%%%%%%%%%%%%%%%%%%%

Najprv dokážem, že pre konečný počet nádob existuje konečný počet 
stavov v ktorých môžu byť. Zoberme si 2 nádoby s kapacitami $k$ a $l$ a 
objemom vody v nich $a$ a $b$. Potom prelievaním viem dostávať iba stavy:
\begin{itemize}
    \item $k$ a $a+b-k$,
    \item $l$ a $a+b-l$,
    \item $0$ a $a+b$,
    \item $a$ a $b$ (ak vodu neprelejem).
\end{itemize}
Avšak keďže nádob je konečný počet, mám konečný počet kapacít aj objemov 
a teda aj rôznych možných súčtov objemov, čo znamená, že aj konečný 
počet stavov.

Potom keďže chcem nájsť najrýchlejší spôsob prelievania, môžem použiť 
prehľadávanie grafu do šírky, pričom vrcholmi budú stavy nádob a 
hrany budú popisovať zmenu stavu, ktorú viem spraviť na 1 preliatie.
(Analogicky cesty grafu na $n$ preliatí.)

Stav v ktorom sa aktuálne nachádzam nech je $S$, pole prejdených stavov 
nazvem $CLOSED$ a frontu (pole stavov na prejdenie) nazvem $OPENED$. 
Na začiatku bude $S$ rovné vstupu, $CLOSED$ prázdne a v $OPENED$ bude 
len $S$. Aloritmus bude vyzerať takto:
\begin{itemize}
    \item Začni cyklus, 
    \item ak je $OPENED$ prázdne, vypíš, že sa prelievaním nedá dostať 
    do cieleného stavu a skonči, 
    \item ináč odstráň prvý prvok z $OPENED$, ulož ho do $S$ a pridaj 
    ho na koniec do $CLOSED$, 
    \item ak sa v $S$ nachádza nádoba, ktorá má aktuálny objem vody taký, 
    aký chceme dostať skonči a vypíš počet preliatí na ktoré sme sa 
    do stavu dostali, ináč pokračuj, 
    \item pre všetky rôzne dvojice nádob zisti do akého stavu sa 
    preliatím vieme dostať (vždy to je z pravidla práve 1 stav z vyššie 
    popísaných) a ak nie je v $OPENED$ ani $CLOSED$, pridaj ho na koniec 
    do $OPENED$, ak je v $OPENED$ pozri sa na počet preliatí ktorým sa 
    doň vieme dostať a ak je ten aktuálny počet menší, prepíš ho, 
    ak je v $CLOSED$ pozri sa na počet preliatí ktorým sa 
    doň vieme dostať a ak je ten aktuálny počet menší, prepíš ho, 
    \item pokračuj na začiatok cyklu.
\end{itemize}
Takto nájdeme najkratší možný počet preliatí na ktorý sa do výsledného 
stavu vieme dostať ak existuje.

%%%%%%%%%%%%%%%%%%%%%%%%%%%%%%%%%%%%%%%%%%%%%%%%%%%%%%%%%%%%%%%%%%%%%%%%
